\documentclass{article}
\usepackage{amsfonts}
\usepackage{amsmath}
\usepackage{hyperref}


\newcommand{\mcF}{\mathcal{F}}
\newcommand{\mcJ}{\mathcal{J}}
\newcommand{\mbF}{\mathbf{F}}
\newcommand{\dmbF}{\delta\mathbf{F}}
\newcommand{\mbI}{\mathbf{I}}
\newcommand{\mbJ}{\mathbf{J}}
\newcommand{\mbK}{\mathbf{K}}
\newcommand{\mbP}{\mathbf{P}}
\newcommand{\mbp}{\mathbf{p}}
\newcommand{\mbX}{\mathbf{X}}
\newcommand{\mbx}{\mathbf{x}}
\newcommand{\mbf}{\mathbf{f}}
\newcommand{\mbq}{\mathbf{q}}
\newcommand{\mbu}{\mathbf{u}}

\begin{document}
	\section{Hexahedral Finite Elements}
	\begin{table}
		\centering
		\begin{tabular}{| r | p{8cm} |}
			\hline
			$\mbF$ & Deformation gradient\\
			\hline
			$\mcJ$ & Displacement function.\\
			\hline
			$\mbJ$ & Derivative of $\mcJ$ with respect to degrees of freedom $\mbq$.\\
			\hline
			$\mbK$ & Stiffness matrix.\\
			\hline
			$N$ & Shape function. \\
			\hline
			$\mbP$ & First Piola-Kirchoff stress. Maps from a normal in reference space to traction in world space.\\
			\hline
			$\Psi$ & Strain energy.\\
			\hline
			$\mbq$ & Degrees of freedom. In this document, it refers to vertex positions of an element.\\
			\hline
			$\mbu$ & Displacement. $\mbx = \mbX+\mbu$.\\
			\hline
			V & Total potential energy. Internal strain energy plus energy from external forces.\\
			\hline
			$\mbx$ & A point in world space.\\
			\hline
			$\mbX$ & A point in reference space.\\
			\hline
		\end{tabular}
		\caption{Notations used in the documentation}
		\label{table:symbols}
	\end{table}
	
We use $\mbX_i$ to denote its vertex positions in its reference space.
	$\mbX_i$ is specified before the simulation and remains constant.
	Following the notation conventions, we use $\mbx_i$ to denote the deformed state
	of the cube.
	
	The elastic behavior of the cube is governed by a non-negative scalar function called the strain energy function $\Psi(\mbx)$.
	At rest shape $\Psi(\mbx)$ takes the minimum value $0$. When forces are applied on vertecies of the cube, the
	cube would deform to minimize the total potential energy defined as
	\begin{equation}
	V=\Psi(\mbx) - \sum_{i=1}^8\mbu_i\cdot \mbf_i,
	\label{eq:pot}
	\end{equation}
	where $\mbf_i$ is a force vector applied to vertex $i$.
	\subsection{Material Model}
	Instead of directly using $\mbx$, 
	we define the strain energy functions in terms of deformation gradients $\mbF_j$
	sampled at a small number of quadrature points $\mbX_j$. The deformation gradients
	depend on $\mbx$.
	\[\Psi(\mbx) = \sum_j \Psi(\mbF_j(\mbx)).\]
	For example, using 3-dimensional
	2-point Gaussian Quadrature rule, we have $8$ sample points in the reference space of the cube:
	$\mbX_j=(\pm\sqrt{\frac{1}{3}}, \pm\sqrt{\frac{1}{3}},\pm\sqrt{\frac{1}{3}})$. Subscript $j$ is used
	denote a quadrature point and $i$ is used to denote a cube vertex.
	
	To compute deformation gradient $\mbF_j$ at $\mbp_j$, we need to know how the volume has deformed
	near $\mbp_j$. Let $\mbu_i=\mbx_i-\mbX_i$ be the displacement for vertex $i$, and $\mbu_j$
	be the displacement for $\mbX_j$.
	We use trilinear interpolation to define displacement $\mbu_j$ by
	\[\mbu_j=\frac{1}{8}\sum_{i=1}^8(1+w_{i,1}\mbX_j^x)(1+w_{i,2}\mbX_j^y)(1+w_{i,3}\mbX_j^z)\mbu_i,\]
	\[
	w=\begin{pmatrix}
	-1 & -1 & -1\\
	-1 & -1 & 1\\
	-1 & 1 & -1\\
	-1 & 1 & 1\\
	1 & -1 & -1\\
	1 & -1 & 1\\
	1 & 1 & -1\\
	1 & 1 & 1
	\end{pmatrix}.
	\]
	\[
	F_j=\frac{d\mbu_j}{d\mbX_j} = \mbI + \sum_{i=1}^8\nabla_{\mbX} N_i(\mbX_j)\mbu_i,
	\]
	\[
	\nabla_{\mbX}N_i(\mbX_j) = \frac{1}{4}
	\begin{pmatrix}
	\frac{w_{i,0}}{\mbX_7^x-\mbX_1^x}(1+w_{i,1}\mbX_j^y)(1+w_{i,2}\mbX_j^z)\\
	\frac{w_{i,1}}{\mbX_7^y-\mbX_1^y}(1+w_{i,0}\mbX_j^x)(1+w_{i,2}\mbX_j^z)\\
	\frac{w_{i,2}}{\mbX_7^z-\mbX_1^z}(1+w_{i,0}\mbX_j^x)(1+w_{i,1}\mbX_j^y)
	\end{pmatrix}^T.
	\]
	A simple and robust material model is called the neo-hookean material
	model and it defines the strain energy function as
	\[
	\Psi(\mbF)=\frac{\mu}{2}(I_1-3) - \mu J + \frac{\lambda}{2}J^2,
	\]
	\[I_1=tr(\mbF^T\mbF),\, J=\log(\det(\mbF)).
	\]
	$\mu$ and $\lambda$ are shear modulus and first Lam\'{e} constant of the material.
\section{Solving for Statics}
	We want to minize the potential energy function in \autoref{eq:pot}.
	An easy first step is to take the derivative of the function and walk in the
	opposite direction of the gradient:
	\begin{equation}
	\frac{\partial V}{\partial \mbx} = \frac{\partial\Psi}{\partial\mbx} - \mbf.
	\end{equation}
	The second term of the gradient tells us the moving vertices along the directions of
	$\mbf$ can reduce the potential energy caused by these forces.
	The first term can be expanded using chain rule
	\begin{align}
	\frac{\partial\Psi}{\partial\mbx_i} &= \sum_j\frac{\partial\Psi}{\partial\mbF_j} \frac{\partial\mbF_j}{\partial\mbx_i}\\
	&=\mbP(\mbF)\nabla N_i(\mbX_j).
	\end{align}
	$\mbP(\mbF)$ is a material specific function known as the first Piola-Kirchoff stress.
	For Neo-hookean model,
	\[\mbP(\mbF) = \mu(\mbF-\mbF^{-T})+\lambda JF^{-T}.\]
	To improve convergence, we use newton's method to minize the potential energy $V$.
	The second derivative of $V$ is called the stiffness matrix $\mbK$.
	$\mbK$ only depends on $\Psi(\mbF)$.
	Taking derivative of $\mbP$ with respect to $\mbF$ results in a third order tensor.
	In a particular direction of $\mbF$ denoted as $\dmbF$, the amount of change
	in $\mbP$ is
	\[
	\delta\mbP(\mbF;\dmbF) = \mu\dmbF+(\mu-\lambda J)\mbF^{-T}\dmbF^T\mbF^{-T}	+\lambda tr(\mbF^{-1}\dmbF)\mbF^{-T}.
	\]

\section{Hierarchical Displacement Field(Under construction)}
$\mbq_c$ and $\mbq_d$ are coarse and detail vertex displacements.
$\mbq_d$ is written in the reference frame of the coarse element.
$\mathbf{X}$ are natural coordinates in an element in its reference configuration.
$\mathbf{u}$ are displacements at $\mathbf{X}$.
\[
\mbu = \mcJ_c(\mbq_c, \mbX + \mcJ_d(\mbq_d,\mbX))
\approx \mcJ_c(\mbq_c, \mbX) + \mbF_c\mcJ_d(\mbq_d,\mbX).
\]
\[
\mcJ(\mbq,\mbX) = \sum_{i=1}^8 N(\mbX)\mbq_i.
\]
The deformation gradient is
\[
\mbF (\mbq,\mbX) = \mbI+\frac{\partial\mcJ(\mbq, \mbX)}{\partial\mbX}=
\mbI+ \sum_i \mbq_i \nabla N^T(\mbX).
\]
\[
\mbF_e = \mbI+\frac{\partial \mbu}{\partial \mbX} 
=\mbI + \frac{\partial \mcJ_c}{\partial\mbX} 
+ \frac{\partial \mcJ_d}{\partial\mbX} + 
\frac{\partial \mcJ_c}{\partial\mbX}\frac{\partial \mcJ_d}{\partial\mbX} 
+ \frac{\partial^2\mcJ_c}{\partial\mbX^2}\mcJ_d
\]
\[=\mbF_c\mbF_d + \frac{\partial^2\mcJ_c}{\partial\mbX^2}\mcJ_d
\approx \mbF_c\mbF_d.
\]

The strain energy can be defined in the same way in terms of $\mbF$.
The problem is that there isn't a unique minimizer given forces and constraints.

Embed quadrature points $\mbX_i$ in fine cubes.
Define strain energy density $\Psi$ at each quadrature point as usual in terms of $\mbF_e(\mbq, \mbX_i)$.
Minimize potential energy
\[
\arg\min_{\mbq} V = \sum_i \Psi( \mbF_e(\mbq, \mbX_i) )
- \mbf_{ext} \cdot \mbq.
\]
Need to distribute stress to $\mbq_c$ and $\mbq_d$ so that we don't double count forces. There are $24$ extra degrees
of freedom from the coarse cube.

Project internal force to the coarse degrees of freedom. The coarse
displacements handle as much stress as possible.
Using chain rule, force on
a coarse vertex $\mbq_i$ is
\[
\mbf_{ci} = \frac{\partial\Psi(\mbF)}{\partial \mbq_i} = 
\sum_{j=1}^{64}w_j\mbP(\mbF_j)\mbF_d^T\nabla N_i(\mbX_j),
\]
where $\mbP=\frac{\partial\Psi}{\partial \mbF}$ is the first Piola-Kirchhoff stress and $w_j$ are quadrature weights.
Similarly for a fine vertex,
\[
\mbf_{di} = 
\sum_{j=1}^{8}w_j\mbP(\mbF_j)\mbF_c\nabla N_i(\mbX_j).
\]

We can project the fine forces into the column space of 
$\mbJ_c$ ($81\times 24$)
to get $f_{dc}$,
the amount of detail forces handled by the coarse vertices.
\[
\mbJ_c=\frac{\partial\mcJ}{\partial\mbq}=\begin{pmatrix}
N_1(\mbX)\\
N_2(\mbX)\\
...\\
N_8(\mbX)
\end{pmatrix}^T.
\]
If we coarsen 8 elements to 1. $\mbJ_c$ is a $81\times 24$ matrix.
\[
\mbf_{c} = (\mbJ_c^T\mbJ_c)^{-1}\mbJ_c^T\mbf_d.
\]
\[
\mbf_{dc} = \mbJ\mbf_c.
\]

The forces left for the fine vertices are 
\[
\mbf_d' = \mbf_d - \mbf_{dc}.
\]
Apply the same idea to split the stiffness
matrix for coarse and fine displacements.
Given the stiffness matrix for the fine vertices $\mbK_d$,
\[
d\mbq_d = \mbK_d\mbf_d.
\]
$\mbK_c=\mbJ_c^T\mbK_d\mbJ_c$.
We can project the displacement to the coarse level to minimize the difference
\[
\|\mbK_c d\mbq_c - \mbK_d\mbf_d\|^2.
\]
\[
\mbq_c = \mbK_c^{-1}\mbJ_c^T\mbK_d\mbf_d.
\]
The amount of detail displacement removed by coarse displacement
is
\[
\mbq_{dc} = \mbJ_c\mbK_c^{-1}\mbJ_c^T\mbK_d\mbf_d.
\]

\section{Continuous}
	\begin{table}
		\centering
		\begin{tabular}{| r | p{8cm} |}
			\hline
			$B$ & Stacked matrix of shape function gradients. $6\times 24$.\\
			\hline
			$C$ & Material parameters stacked in a $6\times 6$ matrix.\\
			\hline
			$f$ & External force.\\
			\hline
			$H$ & Stacked matrix of shape functions. $3\times 24$.\\
			\hline
			$\varepsilon$ & Strain.\\
			\hline
			$n$ & Outward pointing normal.\\
			\hline
			$N$ & Outward pointing normal arranged into a matrix to work with vector representation of stress.\\
			\hline
			$\sigma$ & Cauchy stress. Maps from a normal in the world space to traction in world space.\\
			\hline
			$\hat{\sigma}$ & Guessed/approximated stress.\\
			\hline
			$\tau$ & Test function for strain. Virtual strain.\\
			\hline
			$U$ & Stacked matrix of displacements. $24\times 1$ for a hexahedral element.\\
			\hline
			$u$ & Displacement. \\
			\hline
			$\hat{u}$ & Guessed/approximated displacement.\\
			\hline
			$v$ & Test function for displacement. Virtual displacement.\\
			\hline
		\end{tabular}
		\caption{Notations used in the dgfem sections.}
		\label{table:dgsymbols}
	\end{table}
	Solve
	\[
	-\nabla\sigma(u) = f,
	\]
	where $\sigma(x) = C:\varepsilon(u)$, 
		  $\varepsilon(u,x) = \frac{1}{2}(\nabla u + \nabla u^T)$.
	Auxiliary variable $\sigma = C\varepsilon(u)$.
	Using the symmetry of the stress matrix $\sigma$, one can derive that
	\[
	\nabla v : C \varepsilon(u) = \varepsilon(v) : C : \varepsilon(u).
	\]
	Multiply by test functions $v$ and $\tau$ and integrate by parts in each element $K$,
	\begin{align}
	\int_K \sigma \cdot \tau&=-\int_Ku\nabla\cdot C\tau dx +\int_{\partial K}u \cdot n^T C\tau ds\\
	\int_K \sigma \cdot \nabla v&=\int_K fv +\int_{\partial K}n^T\sigma v.
	\end{align}
	Substitute in numerical fluxes $\hat{u}$ and $\hat{\sigma}$ for boundary integrals.
	\begin{align}
	\int_K \sigma \cdot \tau&=-\int_Ku\nabla\cdot C\tau dx +\int_{\partial K}\hat{u} \cdot  n^T C\tau \label{eq:aux}\\
	\int_K \sigma \cdot \nabla v&=\int_K fv +\int_{\partial K}n^T\hat{\sigma} v .
		\label{eq:force}
	\end{align}
	Integration by parts second time for $\int_Ku\nabla\cdot C\tau dx$ in \autoref{eq:aux}:
	\begin{equation}
	\int_Ku\nabla\cdot C\tau dx=\int_{\partial K}u \cdot n^TC\tau-\int_K\nabla u\cdot C\tau.
	\label{eq:ibp}
	\end{equation}
	Let $\tau=\varepsilon(v)$ and substitute \autoref{eq:ibp} into \autoref{eq:aux}:
	\begin{equation}
	\int_K \sigma \cdot \varepsilon(v) = -\int_{\partial K}u \cdot n^TC\varepsilon(v)+\int_K\nabla u\cdot C \varepsilon(v) +\int_{\partial K}\hat{u} \cdot n^TC\varepsilon(v)
	\label{eq:stress}
	\end{equation}
	Substitute \autoref{eq:stress} into \autoref{eq:force} to get
	\begin{equation}
	\int_K fv + \int_{\partial K}n^T\hat{\sigma}v
	= \int_{\partial K}(\hat{u}-u)\cdot n^TC \varepsilon(v)+\int_K\varepsilon(u) \cdot C\varepsilon(v).
	\end{equation}
		
\section{Discrete}
	\[u(x)=\sum_{i=1}^{8}N_i(x)U_i = H(x)U\]
	\[U = \begin{pmatrix}
	u_{1,x} & u_{1,y} & u_{1,z} & u_{2,x} & u_{2,y} & u_{2,z} & ...\\
	\end{pmatrix}^T(24\times 1)\]
	\[H(x) = \begin{pmatrix}
	  N_1 & 0   & 0   & N_2 & 0   & 0   & ...\\
	  0   & N_1 & 0   & 0   & N_2 & 0   & ...\\
	  0   & 0   & N_1 & 0   & 0   & N_2 & ...\\	  	  
	\end{pmatrix}(3\times 24)\]
	\[B(x) = \begin{pmatrix}
	  \frac{\partial N_1}{\partial x_1} & 0 & 0& ...\\
	  0 & \frac{\partial N_1}{\partial x_2} & 0& ...\\
	  0 & 0 & \frac{\partial N_1}{\partial x_2}& ...\\
 	  \frac{\partial N_1}{\partial x_2} & \frac{\partial N_1}{\partial x_1} & 0& ...\\
  	  0 & \frac{\partial N_1}{\partial x_3} & \frac{\partial N_1}{\partial x_2}& ...\\
  	  \frac{\partial N_1}{\partial x_3} & 0 & \frac{\partial N_1}{\partial x_1}& ...
	\end{pmatrix}(6\times 24)\]
	\[\varepsilon(u,x) = B(x)U\]
	Choose $\hat{\sigma}=C\varepsilon(z)$, $\hat{u}=Z$.
	\[V\int_K B(x)CB(x)U + V\int_{\partial K}B^T(x)C^TN^TH(x)(Z-U)
	=V\int_K H^T(x)f+V\int_{\partial K}H(x)NCB(x)Z.\]
	Remove $V$.
	\[\int_K B(x)CB(x)U + \int_{\partial K}B^T(x)C^TN^TH(x)(Z-U)
	=\int_K H^T(x)f + \int_{\partial K}H(x)NCB(x)Z.\]
	
	\[
	N = \begin{pmatrix}
	n_x & 0   & 0   & n_y & 0   & n_z\\
	0   & n_y & 0   & n_x & n_z & 0 \\
	0   & 0   & n_z & 0   & n_y & n_x  \\		
	\end{pmatrix}(3\times 6)
	\]
	In reference space
	\[KU + \int_{\partial K}P(x)N^TH(x)(Z-U)
	=\int_K H^T(x)f + \int_{\partial K}H(x)NP(x)Z.\]
	Let $T=\int_{\partial K}H(x)NP(x)$, $F = \int_K H^T(x)f$, rearrange
	\[(K-T^T)U=F + (T-T^T)Z.\]
	
\section{Linear Material}
	P(x) = E(x)B(x)
\section{Two Level Embedding}

\end{document}